\documentclass[a4paper,twoside]{article}
\usepackage[T1]{fontenc}
\usepackage[bahasa]{babel}
\usepackage{graphicx}
\usepackage{graphics}
\usepackage{float}
\usepackage[cm]{fullpage}
\pagestyle{myheadings}
\usepackage{etoolbox}
\usepackage{setspace} 
\usepackage{lipsum} 
\setlength{\headsep}{30pt}

\usepackage[inner=2cm,outer=2.5cm,top=2.5cm,bottom=2cm]{geometry} %margin
% \pagestyle{empty}

\makeatletter
\renewcommand{\@maketitle} {\begin{center} {\LARGE \textbf{ \textsc{\@title}} \par} \bigskip {\large \textbf{\textsc{\@author}} }\end{center} }
\renewcommand{\thispagestyle}[1]{}
\markright{\textbf{\textsc{AIF401/AIF402 \textemdash Rencana Kerja Skripsi \textemdash Sem. Genap 2020/2021}}}

\newcommand{\HRule}{\rule{\linewidth}{0.4mm}}
\renewcommand{\baselinestretch}{1}
\setlength{\parindent}{0 pt}
\setlength{\parskip}{6 pt}

\onehalfspacing
 
\begin{document}

\title{\@judultopik}
\author{\nama \textendash \@npm} 

%tulis nama dan NPM anda di sini:
\newcommand{\nama}{Stephen Hadi}
\newcommand{\@npm}{2017730016}
\newcommand{\@judultopik}{Analisa Kekurangan dan Implementasi Perbaikan Perangkat Lunak Bluetape} % Judul/topik anda
\newcommand{\jumpemb}{1} % Jumlah pembimbing, 1 atau 2
\newcommand{\tanggal}{20/09/2020}

% Dokumen hasil template ini harus dicetak bolak-balik !!!!

\maketitle

\pagenumbering{arabic}

\section{Deskripsi}
Bluetape adalah aplikasi website FTIS UNPAR yang berguna untuk memudahkan kegiatan administrasi untuk mahasiswa FTIS.
Bluetape dibuat menggunakan \textit{framework} \textit{Bootstrap} dan \textit{Code Igniter} versi \textit{3}.Dalam aplikasi website ini mahasiswa dapat melihat jadwal dosen, mengajukan permintaan transkrip nilai. Tata usaha dapat mengatur permintaan transkrip nilai yang diajukan oleh mahasiswa.

Pada skripsi ini, \textit{libraries} yang ada pada bluetape akan di \textit{update} menjadi versi yang lebih baru. Juga akan dianalisa kekurangan pada aplikasi ini dan perbaikannya. Ada juga fitur baru yang akan ditambahkan berupa menampilkan statistik manajemen cetak transkrip dan manajemen perubahan kuliah dalam bentuk chart. Fitur tambahan selanjutnya akan dilakukan dengan meminta masukan kepada pengguna bluetape dan mengimplementasi fitur yang bisa dikerjakan.

Implementasi dari perangkat lunak ini akan menggunakan \textit{HTML}, \textit{CSS}, \textit{Javascript} untuk bagian tampilan dengan ditambahkan \textit{framework Bootstrap}, sedangkan bagian \textit{Back-end} akan menggunakan \textit{PHP , MySQL} sedangkan \textit{frameworknya} menggunakan \textit{Code Igniter}.

\section{Rumusan Masalah}
Rumusan masalah yang akan dibahas sebagai berikut:
\begin{itemize}
	\item Bagaimana menentukan pertanyaan survei dan menganalisa implementasi yang memungkinkan?
	\item Bagaimana mencari \textit{bug} yang dilaporkan dalam program? 
	\item Bagaimana mengimplementasi penempatan \textit{chart} agar tidak memberikan informasi yang berlebihan ke pengguna?
	\item Bagaimana melakukan tes setelah memperbarui dan mengganti \textit{libraries} untuk memastikan fungsi dapat berjalan dengan lancar?

\end{itemize}

\section{Tujuan}
Tujuan dari penulisan topik skripsi ini sebagai berikut:
\begin{itemize}
	\item Menambahkan fitur baru yang diinginkan oleh pengguna
	\item Memperbaiki \textit{bug} yang dilaporkan oleh pengguna
	\item Tersedianya statistik manajemen cetak transkrip dan manajemen perubahan kuliah secara visual
	\item Memperbarui dan mengganti \textit{libraries} serta memastikan semua fungsi berjalan dengan lancar.
\end{itemize}

\section{Deskripsi Perangkat Lunak}

Perangkat lunak akhir akan memiliki fitur sebagai berikut:
\begin{itemize}
	\item Memiliki setidaknya satu fitur tambahan yang diinginkan pengguna
	\item Semua mahasiswa dapat melihat jadwal dosen
	\item Tersedianya tampilan chart untuk manajemen cetak transkrip dan manajemen perubahan kuliah
	
\end{itemize}

\section{Detail Pengerjaan Skripsi}

Bagian-bagian pekerjaan skripsi ini adalah sebagai berikut :
	\begin{enumerate}
		\item Memahami dan mengerti cara kerja \textit{framework Code Igniter}
		\item Memahami penggunaan \textit{Framework Bootstrap}.
		\item Memastikan semua fitur berjalan setelah \textit{update libraries} .
		\item Memperbaiki \textit{bug} mahasiswa baru tidak dapat Melihat jadwal dosen
		\item Menganalisa dan merancang fitur chart pada manajemen cetak transkrip dan manajemen perubahan kuliah.
		\item Mengimplementasi fitur chart pada manajemen cetak transkrip dan manajemen perubahan kuliah
		\item Melakukan survei.
		\item Menganalisa dan merancang hasil survei.
		\item Mengimplementasi hasil survei	.	
		\item Menulis dokumen skripsi.
	\end{enumerate}

\section{Rencana Kerja}
Rencana kerja yang akan diambil pada skripsi 1 dan skripsi 2 sebagai berikut:
\begin{table}[H]
	\centering
	\begin{tabular}{|c|c|c|c|c|}
		\hline
		1* & 2*(\%) & 3*(\%) & 4*(\%) & 5*\\
		\hline
		1 & 8 & 8 & 0 & \\
		\hline
		2 & 8 & 8 & 0 & \\
		\hline
		3 & 4 & 4 & 0 & \\
		\hline
		4 & 8 & 8 & 0 & \\
		\hline
		5 & 7 & 7 & 0 &\\
		\hline
		6 & 8 & 8 & 0 & \\
		\hline
		7 & 5 & 5 & 0 & \\
		\hline
		8 & 7 & 0 & 7 & \\
		\hline
		9 & 13 & 0 & 13 & \\
		\hline
		10 & 30 & 5 & 25 & Menyelesaikan Bab 1 - Bab 3 pada skripsi 1 dan kurangnya pada skripsi 2\\
		\hline
		Total & 100 & 53 & 47 &  \\
		\hline
		
	\end{tabular}
\end{table}	

Keterangan (*):
\begin{enumerate}
	\item : Bagian pengerjaan skripsi dapat dilihat pada bagian 5 
	\item : Persentase total
	\item : Persentase yang akan dikerjakan pada skripsi 1
	\item : Persentase yang akan dikerjakan pada skripsi 2
	\item : Keterangan singkat apa yang akan dikerjakan pada skripsi 1 dan skripsi 2
\end{enumerate}


\vspace{1cm}
\centering Bandung, \tanggal\\
\vspace{2cm} \nama \\ 
\vspace{1cm}


Menyetujui, \\
\ifdefstring{\jumpemb}{2}{
\vspace{1.5cm}
\begin{centering} Menyetujui,\\ \end{centering} \vspace{0.75cm}
\begin{minipage}[b]{0.45\linewidth}
% \centering Bandung, \makebox[0.5cm]{\hrulefill}/\makebox[0.5cm]{\hrulefill}/2013 \\
\vspace{2cm} Nama: \makebox[3cm]{\hrulefill}\\ Pembimbing Utama
\end{minipage} \hspace{0.5cm}
\begin{minipage}[b]{0.45\linewidth}
% \centering Bandung, \makebox[0.5cm]{\hrulefill}/\makebox[0.5cm]{\hrulefill}/2013\\
\vspace{2cm} Nama: \makebox[3cm]{\hrulefill}\\ Pembimbing Pendamping
\end{minipage}
\vspace{0.5cm}
}{
% \centering Bandung, \makebox[0.5cm]{\hrulefill}/\makebox[0.5cm]{\hrulefill}/2013\\
\vspace{2cm} Nama: \makebox[3cm]{\hrulefill}\\ Pembimbing Tunggal
}
\end{document}

