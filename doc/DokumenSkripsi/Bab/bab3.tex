\chapter{Analisis}
\section{Analisis Kebutuhan Perangkat Lunak \textit{Bluetape}}
Pengerjaan untuk menganalisis \textit{bluetape} dilakukan dengan melakukan survei dengan pengguna--pengguna aplikasi tersebut. Survei dilakukan melalui \textit{google form} dan \textit{link} dari survei tersebut dibagikan melalui \textit{mailing list} dosen dan grup informatika UNPAR angkatan 2017. Pertanyaan pada survei tersebut cukup singkat. Pertanyaan hanya menanyakan nama dan tipe pengguna \textit{bluetape} dan disediakan suatu text field untuk memasukkan \textit{feedback} dari pengguna. Adapun \textit{feedback} yang telah terkumpulkan tersedia pada tabel \ref{tab:analisis pengguna}.

\begin{table}[H]
	\centering
	\label{tab:analisis pengguna}
	\resizebox{\textwidth}{!}{
	\begin{tabular}{|l|l|l|l|}
	\hline
	No & Deskripsi & \textit{Issue Number}  & Status \\ \hline
	1 & Memperbaiki ekspor ke xls pada EntriJadwalDosen & \#1  & Akan diimplementasi\\ \hline
	2 & Memperbaharui \textit{google api} dan \textit{phpspreadsheet} & \#2  & Akan diimplementasi \\ \hline
	3 & Tampilan \textit{chart} pada ManajemenCetakTranskrip & \#3  & Akan diimplementasi\\ \hline
	4 & Tampilan \textit{chart} pada ManajemenPerubahanKuliah & \#4  & Akan diimplementasi\\ \hline
	5 & \makecell[l]{Mahasiswa dengan format NPM baru \\ tidak dapat login dan LihatJadwalDosen} & \#5  & Akan diimplementasi\\ \hline
	6 & \makecell[l]{Kolom pada EntriJadwalDosen dan \\ LihatJadwalDosen tidak seragam}  & \#6 & Akan diimplementasi\\ \hline
	7 & Fungsi Tab pada LihatJadwalDosen tidak berfungsi & \#7 & Akan diimplementasi\\ \hline
	8 & \makecell[l]{Pengelompokkan rekap perubahan jadwal\\ pada mata kuliah yang sama} & \#8 & \\ \hline
	9 & Forum Q\&A pada \textit{bluetape} & \#9 & \\ \hline
	10 & \textit{Scheduling} matakuliah pada googlemeet/zoom & \#10 & Tidak diimplementasi\\ \hline
	11 & Mengubah atau membatalkan permohonan & \#11 & Akan diimplementasi\\ \hline
	12 & Menambahkan jam kuliah selesai di perubahan kelas & \#12 & Akan diimplementasi\\ \hline
	13 & Memperbaiki form dan link yang tidak aktif & \#13 & Tidak diimplementasi\\ \hline
	14 & Mengintegrasi \textit{bluetape} dengan SSO UNPAR & \#14 & Tidak diimplementasi\\ \hline
	15 & Menambah list permintaan pada \textit{bluetape} & \#15 & Tidak diimplementasi\\ \hline
	16 & Pengajuan surat keterangan aktif kuliah secara online & \#16 & \\ \hline
	17 & \makecell[l]{Notifikasi email untuk mahasiswa \\ jika permintaan sudah diselesaikan} & \#17 & Akan diimplementasi\\ \hline
	18 & Dapat melihat profil mahasiswa & \#18 & Tidak diimplementasi\\ \hline
	19 & Halaman histori dan request transkrip terpisah & \#19 & \\ \hline
	20 & Fitur bahasa indonesia dan inggris & \#20 & \\ \hline
	21 & Pagination tidak terstyle dengan baik & \#22 & Akan diimplementasi\\ \hline
	22 & Format Datetimepicker tidak konsisten &\#23 & Akan diimplementasi\\ \hline
	
	
	\end{tabular}}
	\caption{Tabel analsis kebutuhan pengguna perangkat \textit{bluetape}}
\end{table}


\section{Bluetape}

\textit{Bluetape} adalah aplikasi + framework untuk membuat urusan-urusan paper-based di FTIS UNPAR menjadi paperless.\footnote{\url{https://github.com/ftisunpar/BlueTape}}

\subsection{Instalasi}

\subsection{Struktur Bluetape}

\subsection{Pengaturan Dasar Bluetape}

\subsection{Database}