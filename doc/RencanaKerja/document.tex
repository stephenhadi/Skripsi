\documentclass[a4paper,twoside]{article}
\usepackage[T1]{fontenc}
\usepackage[bahasa]{babel}
\usepackage{graphicx}
\usepackage{graphics}
\usepackage{float}
\usepackage[cm]{fullpage}
\pagestyle{myheadings}
\usepackage{etoolbox}
\usepackage{setspace} 
\usepackage{lipsum} 
\setlength{\headsep}{30pt}

\usepackage[inner=2cm,outer=2.5cm,top=2.5cm,bottom=2cm]{geometry} %margin
% \pagestyle{empty}

\makeatletter
\renewcommand{\@maketitle} {\begin{center} {\LARGE \textbf{ \textsc{\@title}} \par} \bigskip {\large \textbf{\textsc{\@author}} }\end{center} }
\renewcommand{\thispagestyle}[1]{}
\markright{\textbf{\textsc{AIF401/AIF402 \textemdash Rencana Kerja Skripsi \textemdash Sem. Genap 2020/2021}}}

\newcommand{\HRule}{\rule{\linewidth}{0.4mm}}
\renewcommand{\baselinestretch}{1}
\setlength{\parindent}{0 pt}
\setlength{\parskip}{6 pt}

\onehalfspacing
 
\begin{document}

\title{\@judultopik}
\author{\nama \textendash \@npm} 

%tulis nama dan NPM anda di sini:
\newcommand{\nama}{Stephen Hadi}
\newcommand{\@npm}{2017730016}
\newcommand{\@judultopik}{Analisa dan Implementasi Perbaikan Perangkat Lunak Bluetape} % Judul/topik anda
\newcommand{\jumpemb}{1} % Jumlah pembimbing, 1 atau 2
\newcommand{\tanggal}{05/10/2020}

% Dokumen hasil template ini harus dicetak bolak-balik !!!!

\maketitle

\pagenumbering{arabic}

\section{Deskripsi}
Bluetape adalah aplikasi website FTIS UNPAR yang berguna untuk memudahkan kegiatan administrasi untuk mahasiswa FTIS.
Bluetape dibuat menggunakan \textit{framework} \textit{Bootstrap} dan \textit{Code Igniter} versi \textit{3}. Dalam aplikasi website ini mahasiswa dapat melihat jadwal dosen, mengajukan permintaan transkrip nilai.Tata usaha dapat menerima/menolak permintaan transkrip nilai yang diajukan oleh mahasiswa.

Pada skripsi ini, \textit{libraries} yang ada pada bluetape akan di \textit{update} menjadi versi yang lebih baru. Juga akan memperbaiki fitur--fitur yang tidak berjalan yang dilaporkan oleh pengguna. Ada juga fitur baru yang akan ditambahkan berupa menampilkan statistik manajemen cetak transkrip dan manajemen perubahan kuliah dalam bentuk chart. Fitur tambahan selanjutnya akan dilakukan dengan meminta masukan kepada pengguna bluetape dan melakukan analisa dan percangan dilanjutkan dengan mengimplementasi fitur yang bisa dikerjakan.

Implementasi dari perangkat lunak ini akan menggunakan \textit{HTML}, \textit{CSS}, \textit{Javascript} untuk bagian tampilan dengan ditambahkan \textit{framework Bootstrap}, sedangkan bagian \textit{Back-end} akan menggunakan \textit{PHP , MYSQL} sedangkan \textit{frameworknya} menggunakan \textit{Code Igniter}.

\section{Rumusan Masalah}
Rumusan masalah yang akan dibahas sebagai berikut:
\begin{itemize}
	\item Apa sajakah kebutuhan yang diinginkan oleh pengguna bluetape?
	\item Bagaimana menganalisa, merancang, dan mengimplementasi \textit{feedback--feedback} dari pengguna?
	\item Bagaimana melakukan pengujian setelah mengimplementasi \textit{feedback--feedback} dari pengguna?

\end{itemize}

\section{Tujuan}
Tujuan dari penulisan topik skripsi ini sebagai berikut:
\begin{itemize}
	\item Mendapatkan \textit{feedback--feedback} dari pengguna terkait kebutuhan yang diinginkan
	\item Menganalisa, merancang, dan mengimplementasi \textit{feedback--feedback} dari pengguna
	\item Melakukan pengujian setelah mengimplementasi semua \textit{feedback} yang memungkinkan untuk diimplementasi
\end{itemize}


\newpage
\section{Deskripsi Perangkat Lunak}

Perangkat lunak akhir akan memiliki fitur sebagai berikut:
\begin{itemize}
	\item Memiliki fitur--fitur sesuai dengan kebutuhan pengguna yang telah didiskusikan dengan pembimbing
	\item fitur--fitur sebelumnya tidak hilang atau tidak berfungsi 
	
\end{itemize}

\section{Detail Pengerjaan Skripsi}

Bagian-bagian pekerjaan skripsi ini adalah sebagai berikut :
	\begin{enumerate}
		\item Memahami dan mengerti cara kerja \textit{framework Code Igniter} dan \textit{Bootstrap}.
		\item Melakukan survei ke mahasiswa, tata usaha, dan dosen untuk memahami kebutuhan pengguna.
		\item Menganalisa fitur kebutuhan bersama pembimbing.
		\item Merancang dan mengimplementasi hasil survei.	
		\item Melakukan pengujian ke pengguna bluetape
		\item Menulis dokumen skripsi.
	\end{enumerate}

\section{Rencana Kerja}


Rincian capaian yang direncanakan di Skripsi 1 adalah sebagai berikut:
\begin{enumerate}
	\item Memahami dan mengerti cara kerja framework \textit{Code Igniter} dan \textit{Bootstrap}.
	\item Menganalisa fitur kebutuhan bersama pembimbing
	\item Melakukan survei ke mahasiswa, tata usaha, dan dosen untuk memahami kebutuhan pengguna.
	\item Merancang dan mengimplementasi hasil survei.
	\item Menulis dokumen skripsi.
\end{enumerate}

Sedangkan yang akan diselesaikan di Skripsi 2 adalah sebagai berikut:
\begin{enumerate}
	\item Melakukan pengujian fungsional dan eksperimental selama 1 semester. 
	\item Menyelesaikan dokumen skripsi.
\end{enumerate}


\newpage
\vspace{1cm}
\centering Bandung, \tanggal\\
\vspace{2cm} \nama \\ 
\vspace{1cm}


Menyetujui, \\
\ifdefstring{\jumpemb}{2}{
\vspace{1.5cm}
\begin{centering} Menyetujui,\\ \end{centering} \vspace{0.75cm}
\begin{minipage}[b]{0.45\linewidth}
% \centering Bandung, \makebox[0.5cm]{\hrulefill}/\makebox[0.5cm]{\hrulefill}/2013 \\
\vspace{2cm} Nama: \makebox[3cm]{\hrulefill}\\ Pembimbing Utama
\end{minipage} \hspace{0.5cm}
\begin{minipage}[b]{0.45\linewidth}
% \centering Bandung, \makebox[0.5cm]{\hrulefill}/\makebox[0.5cm]{\hrulefill}/2013\\
\vspace{2cm} Nama: \makebox[3cm]{\hrulefill}\\ Pembimbing Pendamping
\end{minipage}
\vspace{0.5cm}
}{
% \centering Bandung, \makebox[0.5cm]{\hrulefill}/\makebox[0.5cm]{\hrulefill}/2013\\
\vspace{2cm} Nama: \makebox[3cm]{\hrulefill}\\ Pembimbing Tunggal
}
\end{document}

