\chapter{Analisis}
\section{Analisis Kebutuhan Perangkat Lunak \textit{Bluetape}}
Pengerjaan untuk menganalisis \textit{bluetape} dilakukan dengan melakukan survei dengan pengguna--pengguna aplikasi tersebut. Survei dilakukan melalui \textit{google form} dan \textit{link} dari survei tersebut dibagikan melalui \textit{mailing list} dosen dan grup informatika UNPAR angkatan 2017. Pertanyaan pada survei tersebut cukup singkat. Pertanyaan hanya menanyakan nama dan tipe pengguna \textit{bluetape} dan disediakan suatu text field untuk memasukkan \textit{feedback} dari pengguna. Adapun \textit{feedback} yang telah terkumpulkan tersedia pada tabel \ref{tab:analisis pengguna}.

\begin{table}[H]
	\centering
	\label{tab:analisis pengguna}
	\resizebox{\textwidth}{!}{
	\begin{tabular}{|l|l|l|l|}
	\hline
	No & Deskripsi & \textit{Issue Number}  & Status \\ \hline
	1 & \makecell[l]{Fitur "Expor ke XLS" pada menu \\ EntriJadwalDosen  menghasilkan file \textit{corrupt}} & \#1  & Akan diimplementasi\\ \hline
	2 & \textit{Update google api / phpspreadsheet}  & \#2  & Akan diimplementasi \\ \hline
	3 & Fitur \textit{chart} pada ManajemenCetakTranskrip & \#3  & Akan diimplementasi\\ \hline
	4 & Fitur \textit{chart} pada ManajemenPerubahanKuliah & \#4  & Akan diimplementasi\\ \hline
	5 & \makecell[l]{Mahasiswa dengan NPM baru \\ tidak dapat login dan LihatJadwalDosen} & \#5  & Akan diimplementasi\\ \hline
	6 & \makecell[l]{Kolom pada EntriJadwalDosen dan \\ LihatJadwalDosen tidak seragam}  & \#6 & Akan diimplementasi\\ \hline
	7 & Fungsi Tab pada LihatJadwalDosen tidak berfungsi & \#7 & Akan diimplementasi\\ \hline
	8 & \makecell[l]{Pengelompokkan rekap perubahan jadwal\\ pada mata kuliah yang sama} & \#8 & \\ \hline
	9 & Menyediakan forum Q\&A pada \textit{bluetape} & \#9 & \\ \hline
	10 & \textit{Scheduling} matakuliah pada googlemeet/zoom & \#10 & Tidak diimplementasi\\ \hline
	11 & Mengubah atau membatalkan permohonan & \#11 & Akan diimplementasi\\ \hline
	12 & Menambahkan jam kuliah selesai di perubahan kelas & \#12 & Akan diimplementasi\\ \hline
	13 & Memperbaiki form dan link yang tidak aktif & \#13 & Tidak diimplementasi\\ \hline
	14 & Mengintegrasi \textit{bluetape} dengan SSO UNPAR & \#14 & Tidak diimplementasi\\ \hline
	15 & Menambah list permintaan pada \textit{bluetape} & \#15 & Tidak diimplementasi\\ \hline
	16 & Pengajuan surat keterangan aktif kuliah secara online & \#16 & \\ \hline
	17 & \makecell[l]{Notifikasi email untuk mahasiswa \\ jika permintaan sudah diselesaikan} & \#17 & Akan diimplementasi\\ \hline
	18 & Dapat melihat profil mahasiswa & \#18 & Tidak diimplementasi\\ \hline
	19 & Halaman histori dan request transkrip terpisah & \#19 & \\ \hline
	20 & Fitur bahasa indonesia dan inggris & \#20 & \\ \hline
	21 & Pagination tidak terstyle dengan baik & \#22 & Akan diimplementasi\\ \hline
	22 & Format Datetimepicker tidak konsisten &\#23 & Akan diimplementasi\\ \hline
	
	
	\end{tabular}}
	\caption{Tabel analsis kebutuhan pengguna perangkat \textit{bluetape}}
\end{table}

Pada tabel \ref{tab:analisis pengguna} terdapat beberapa kolom yaitu:
\begin{itemize}
	\item Deskripsi \\
	Deskripsi adalah judul dari \textit{issue} yang berasal dari \textit{github}.
	\item \textit{Issue Number} \\ \textit{Issue number} adalah nomor \textit{issue} yang berasal dari \textit{github}. \textit{Issue number} yang dibuat adalah berdasarkan \textit{issue number} pada \textit{repository} yang telah di \textit{fork}.
	\item Status \\
	Kolom status menjelaskan apakah \textit{feedback} tersebut akan dikerjakan atau tidak.
\end{itemize}

\subsection{Fitur "Expor ke XLS" pada Menu EntriJadwalDosen Menghasilkan File \\ \textit{Corrupt}}
\begin{figure}[H]
	\centering
	\includegraphics[scale=0.3]{exporkexls} 
	\caption{Antarmuka halaman \texttt{EntriJadwalDosen}}
	\label{fig:EntriJadwalDosen} 
\end{figure}

Fitur ini terdapat pada halaman \texttt{EntriJadwalDosen} yang dapat digunakan untuk memasukan atau merubah jadwal dosen. Dapat dilihat pada gambar \ref{fig:EntriJadwalDosen} terdapat sebuah fitur untuk melakukan \textit{expor ke xls}. Dari laporan pengguna fitur tersebut akan menghasilkan file \texttt{.xls} yang \textit{corrupt}. 

Fitur tersebut akan dikerjakan dan diperbaiki. Tetapi \textit{error} file \textit{corrupt} tidak dapat direproduksi setelah melakukan \textit{update} pada \ref{sub:update google api} dan melakukan perubahan dari \textit{phpexcel} menjadi \textit{phpspreadsheet}. 

\subsection{\textit{Update google api/phpspreadsheet}}
\textit{Bluetape} saat ini menggunakan \textit{google/apiclient} 1.0 dan \textit{phpexcel}. Salah satu permintaan dari pengguna/pengurus \textit{bluetape} untuk merubah \textit{google/apiclient} 1.0 menjadi \textit{google/apiclient} 2.4.0 dan \textit{phpexcel} menjadi \textit{phpspreadsheet} v1.11.0. 

Fitur ini akan dikerjakan dan solusi yang ditawarkan untuk \textit{issue} ini adalah merubah \texttt{composer.json} dan melakukan \texttt{composer update}.
\label{sub:update google api}

\subsection{Fitur \textit{chart} pada ManajemenCetakTranskrip}
\subsection{Fitur \textit{chart} pada ManajemenPerubahanKuliah}
\subsection{Mahasiswa dengan NPM Baru Tidak Dapat \textit{Login} dan LihatJadwalDosen}
\subsection{Kolom pada EntriJadwalDosen dan LihatJadwalDosen Tidak Seragam} 
Pada halaman \texttt{EntriJadwalDosen} dan \texttt{LihatJadwalDosen} terdapat tabel seperti pada gambar \ref{fig:EntriJadwalDosen}. Ukuran tabel tersebut dapat berubah sesuai dengan ukuran tulisan di dalamnya sehingga kolom dengan tulisan yang banyak tidak seragam dengan kolom yang tulisannya sedikit.

Fitur ini akan diimplementasi. Solusi yang ditawarkan adalah dengan memasang \textit{fixed width} pada setiap kolom.

\subsection{Fungsi tab pada LihatJadwalDosen tidak berfungsi}
\subsection{Pengelompokkan Rekap Perubahan Jadwal pada Mata Kuliah yang Sama}
\subsection{Menyediakan forum Q\&A pada \textit{bluetape}}
\subsection{\textit{Scheduling} Matakuliah pada \textit{googlemeet/zoom}}
\textit{Issue} ini meminta agar \textit{bluetape} melakukan \textit{auto include} jadwal kuliah ke calendar, lalu \textit{autocreate} jadwal kuliah berdasarkan jadwal tersebut di \textit{zoom/meet} selama 1 semester, dan \textit{Request} jadwal bertemu dari mahasiswa kepada dosen pembimbing, kemudian \textit{auto schedule zoom/meet}.

Fitur ini tidak akan diimplementasi pada skripsi ini, karena permintaan tersebut masuk ke dalam ranah Biro Teknologi Informasi (BTI) dan saat ini \textit{bluetape} tidak memiliki kemampuan untuk integrasi jadwal kuliah dari BTI.

\subsection{Mengubah atau Membatalkan Permohonan}
\subsection{Menambahkan Jam Kuliah Selesai di Perubahan Kelas}

\subsection{Memperbaiki \textit{form} dan \textit{link} yang Tidak Aktif}
Masukan ini meminta perbaikan terhadap fitur yang ada: 
\begin{itemize}
	\item \textit{form} atau link seperti no surat, daftar ruang, dan peraturan terus diupdate.
	\item Dibuat lebih sistematis karena tampilannya tidak seragam.
	\item Dibuat lebih interaktif seperti peminjaman ruang.
\end{itemize}   

Masukan ini tidak akan diimplementasi, karena sepertinya fitur tersebut di luar sistem \textit{bluetape}. Seperti no surat, daftar ruang, dan peraturan.

\subsection{Mengintegrasi \textit{bluetape} dengan SSO UNPAR}
Masukan ini meminta fitur tambahan agar \textit{bluetape} terintegrasi secara penuh dengan \textit{studentportal}\footnote{\url{https://studentportal.unpar.ac.id/}}. Sehingga \textit{bluetape} dapat diakses lewat \textit{studentportal} dan \textit{bluetape} dapat \textit{login} menggunakan SSO.

Fitur masukan ini tidak akan diimplementasi karena diperlukan koordinasi yang kuat dengan BTI dan \textit{Bluetape} untuk sementara ini tidak diperuntukan untuk pengguna diluar FTIS.

\subsection{Menambah \textit{list} Permintaan pada \textit{Bluetape}}
Masukan ini meminta agar \textit{bluetape} menambah \textit{list} permintaan. Masukan ini tidak dikerjakan karena pengguna tersebut kurang spesifik dalam menjelaskan kebutuhannya.

\subsection{Pengajuan surat keterangan aktif kuliah secara online}
\subsection{Dapat melihat profil }

\section{Bluetape}

\textit{Bluetape} adalah aplikasi + framework untuk membuat urusan-urusan paper-based di FTIS UNPAR menjadi paperless.\footnote{\url{https://github.com/ftisunpar/BlueTape}}

\subsection{Instalasi}

\subsection{Struktur Bluetape}

\subsection{Pengaturan Dasar Bluetape}

\subsection{Database}